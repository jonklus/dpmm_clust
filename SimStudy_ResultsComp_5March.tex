% Options for packages loaded elsewhere
\PassOptionsToPackage{unicode}{hyperref}
\PassOptionsToPackage{hyphens}{url}
%
\documentclass[
]{article}
\usepackage{amsmath,amssymb}
\usepackage{lmodern}
\usepackage{iftex}
\ifPDFTeX
  \usepackage[T1]{fontenc}
  \usepackage[utf8]{inputenc}
  \usepackage{textcomp} % provide euro and other symbols
\else % if luatex or xetex
  \usepackage{unicode-math}
  \defaultfontfeatures{Scale=MatchLowercase}
  \defaultfontfeatures[\rmfamily]{Ligatures=TeX,Scale=1}
\fi
% Use upquote if available, for straight quotes in verbatim environments
\IfFileExists{upquote.sty}{\usepackage{upquote}}{}
\IfFileExists{microtype.sty}{% use microtype if available
  \usepackage[]{microtype}
  \UseMicrotypeSet[protrusion]{basicmath} % disable protrusion for tt fonts
}{}
\makeatletter
\@ifundefined{KOMAClassName}{% if non-KOMA class
  \IfFileExists{parskip.sty}{%
    \usepackage{parskip}
  }{% else
    \setlength{\parindent}{0pt}
    \setlength{\parskip}{6pt plus 2pt minus 1pt}}
}{% if KOMA class
  \KOMAoptions{parskip=half}}
\makeatother
\usepackage{xcolor}
\usepackage[margin=1in]{geometry}
\usepackage{graphicx}
\makeatletter
\def\maxwidth{\ifdim\Gin@nat@width>\linewidth\linewidth\else\Gin@nat@width\fi}
\def\maxheight{\ifdim\Gin@nat@height>\textheight\textheight\else\Gin@nat@height\fi}
\makeatother
% Scale images if necessary, so that they will not overflow the page
% margins by default, and it is still possible to overwrite the defaults
% using explicit options in \includegraphics[width, height, ...]{}
\setkeys{Gin}{width=\maxwidth,height=\maxheight,keepaspectratio}
% Set default figure placement to htbp
\makeatletter
\def\fps@figure{htbp}
\makeatother
\setlength{\emergencystretch}{3em} % prevent overfull lines
\providecommand{\tightlist}{%
  \setlength{\itemsep}{0pt}\setlength{\parskip}{0pt}}
\setcounter{secnumdepth}{-\maxdimen} % remove section numbering
\usepackage{booktabs}
\usepackage{longtable}
\usepackage{array}
\usepackage{multirow}
\usepackage{wrapfig}
\usepackage{float}
\usepackage{colortbl}
\usepackage{pdflscape}
\usepackage{tabu}
\usepackage{threeparttable}
\usepackage{threeparttablex}
\usepackage[normalem]{ulem}
\usepackage{makecell}
\usepackage{xcolor}
\ifLuaTeX
  \usepackage{selnolig}  % disable illegal ligatures
\fi
\IfFileExists{bookmark.sty}{\usepackage{bookmark}}{\usepackage{hyperref}}
\IfFileExists{xurl.sty}{\usepackage{xurl}}{} % add URL line breaks if available
\urlstyle{same} % disable monospaced font for URLs
\hypersetup{
  pdftitle={Model Spec Comparison},
  pdfauthor={Jonathan Klus},
  hidelinks,
  pdfcreator={LaTeX via pandoc}}

\title{Model Spec Comparison}
\author{Jonathan Klus}
\date{2024-03-05}

\begin{document}
\maketitle

\hypertarget{standard}{%
\subsection{Standard}\label{standard}}

\begin{itemize}
\tightlist
\item
  All diagonal models have same prior setup (probably not a good idea)
\item
  \(\sigma^2 \sim IG(a=1, b=50)\)
\item
  For invWish prior, \(\Lambda_0 = diag(15)\)
\end{itemize}

\hypertarget{adj-rand-index}{%
\subsubsection{Adj Rand Index}\label{adj-rand-index}}

\begin{verbatim}
## `summarise()` has grouped output by 'Model', 'Scenario', 'SM'. You can override
## using the `.groups` argument.
\end{verbatim}

\begin{verbatim}
## # A tibble: 6 x 8
## # Groups:   Model, SM [6]
##   Model   SM     `30_3close` `30_3wellsep` `100_3close` 100_3w~1 300_3~2 300_3~3
##   <chr>   <chr>        <dbl>         <dbl>        <dbl>    <dbl>   <dbl>   <dbl>
## 1 conjDEE noSM          0.26          0.43         0.09     0.23   NA      NA   
## 2 conjDEE withSM        0.26          0.43         0.08     0.23   NA      NA   
## 3 conjDEV noSM          0.69          0.98         0.87     0.87    0.9     0.92
## 4 conjDEV withSM        0.04          0.97         0.09     0.97    0.84    0.98
## 5 conjUVV noSM          0.72          0.94         0.77     0.79    0.83    0.8 
## 6 conjUVV withSM        0.62          0.85         0.75     0.85    0.9     0.94
## # ... with abbreviated variable names 1: `100_3wellsep`, 2: `300_3close`,
## #   3: `300_3wellsep`
\end{verbatim}

\hypertarget{kl-divergence}{%
\subsubsection{KL divergence}\label{kl-divergence}}

\begin{verbatim}
## `summarise()` has grouped output by 'Model', 'Scenario', 'SM'. You can override
## using the `.groups` argument.
\end{verbatim}

\begin{verbatim}
## # A tibble: 6 x 8
## # Groups:   Model, SM [6]
##   Model   SM     `30_3close` `30_3wellsep` `100_3close` 100_3w~1 300_3~2 300_3~3
##   <chr>   <chr>        <dbl>         <dbl>        <dbl>    <dbl>   <dbl>   <dbl>
## 1 conjDEE noSM             0             0            0        0      NA      NA
## 2 conjDEE withSM           0             0            0        0      NA      NA
## 3 conjDEV noSM             0             0            0        0       0       0
## 4 conjDEV withSM           0             0            0        0       0       0
## 5 conjUVV noSM             0             0            0        0       0       0
## 6 conjUVV withSM           0             0            0        0       0       0
## # ... with abbreviated variable names 1: `100_3wellsep`, 2: `300_3close`,
## #   3: `300_3wellsep`
\end{verbatim}

\hypertarget{alternative-1-for-close-together}{%
\subsection{Alternative 1 for close
together}\label{alternative-1-for-close-together}}

\begin{itemize}
\tightlist
\item
  \(\sigma^2 \sim IG(a=1, b=25)\)
\item
  For invWish prior, \(\Lambda_0 = diag(8)\)
\end{itemize}

\hypertarget{adj-rand-index-1}{%
\subsubsection{Adj Rand Index}\label{adj-rand-index-1}}

\begin{verbatim}
## `summarise()` has grouped output by 'Model', 'Scenario', 'SM'. You can override
## using the `.groups` argument.
\end{verbatim}

\begin{verbatim}
## # A tibble: 4 x 4
## # Groups:   Model, SM [4]
##   Model   SM     `30_3close` `100_3close`
##   <chr>   <chr>        <dbl>        <dbl>
## 1 conjDEV noSM             1            1
## 2 conjDEV withSM           0            1
## 3 conjUVV noSM             1            1
## 4 conjUVV withSM           1            1
\end{verbatim}

\hypertarget{kl-divergence-1}{%
\subsubsection{KL divergence}\label{kl-divergence-1}}

\begin{verbatim}
## `summarise()` has grouped output by 'Model', 'Scenario', 'SM'. You can override
## using the `.groups` argument.
\end{verbatim}

\begin{verbatim}
## # A tibble: 4 x 4
## # Groups:   Model, SM [4]
##   Model   SM     `30_3close` `100_3close`
##   <chr>   <chr>        <dbl>        <dbl>
## 1 conjDEV noSM             0            0
## 2 conjDEV withSM           0            0
## 3 conjUVV noSM             0            0
## 4 conjUVV withSM           0            0
\end{verbatim}

\hypertarget{alternative-2-for-close-together}{%
\subsection{Alternative 2 for close
together}\label{alternative-2-for-close-together}}

\begin{itemize}
\tightlist
\item
  \(\sigma^2 \sim IG(a=1, b=50)\)
\end{itemize}

\hypertarget{adj-rand-index-2}{%
\subsubsection{Adj Rand Index}\label{adj-rand-index-2}}

\begin{verbatim}
## `summarise()` has grouped output by 'Model', 'Scenario', 'SM'. You can override
## using the `.groups` argument.
\end{verbatim}

\begin{verbatim}
## # A tibble: 1 x 4
## # Groups:   Model, SM [1]
##   Model   SM     `30_3close` `100_3close`
##   <chr>   <chr>        <dbl>        <dbl>
## 1 conjDEV withSM           1            1
\end{verbatim}

\hypertarget{kl-divergence-2}{%
\subsubsection{KL divergence}\label{kl-divergence-2}}

\begin{verbatim}
## `summarise()` has grouped output by 'Model', 'Scenario', 'SM'. You can override
## using the `.groups` argument.
\end{verbatim}

\begin{verbatim}
## # A tibble: 1 x 4
## # Groups:   Model, SM [1]
##   Model   SM     `30_3close` `100_3close`
##   <chr>   <chr>        <dbl>        <dbl>
## 1 conjDEV withSM           0            0
\end{verbatim}

\hypertarget{alternative-3-for-close-together---hyperprior}{%
\subsection{Alternative 3 for close together -
hyperprior}\label{alternative-3-for-close-together---hyperprior}}

\begin{itemize}
\tightlist
\item
  \(\sigma^2 \sim IG(a=1, b)\)
\item
  \(b \sim Ga(g = 1, h = 5)\)
\end{itemize}

\hypertarget{adj-rand-index-3}{%
\subsubsection{Adj Rand Index}\label{adj-rand-index-3}}

\begin{verbatim}
## `summarise()` has grouped output by 'Model', 'Scenario', 'SM'. You can override
## using the `.groups` argument.
\end{verbatim}

\begin{verbatim}
## # A tibble: 1 x 4
## # Groups:   Model, SM [1]
##   Model   SM     `30_3close` `100_3close`
##   <chr>   <chr>        <dbl>        <dbl>
## 1 conjDEV withSM           1            0
\end{verbatim}

\hypertarget{kl-divergence-3}{%
\subsubsection{KL divergence}\label{kl-divergence-3}}

\begin{verbatim}
## `summarise()` has grouped output by 'Model', 'Scenario', 'SM'. You can override
## using the `.groups` argument.
\end{verbatim}

\begin{verbatim}
## # A tibble: 1 x 4
## # Groups:   Model, SM [1]
##   Model   SM     `30_3close` `100_3close`
##   <chr>   <chr>        <dbl>        <dbl>
## 1 conjDEV withSM           0            0
\end{verbatim}

\end{document}
